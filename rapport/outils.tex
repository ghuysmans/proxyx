\section{Outils}
\subsection{Dépendances}
\begin{itemize}
\item Les bibliothèques OpenSSL et Lua;
\item Un compilateur C++ car CMake en a besoin.
\end{itemize}

\subsection{Documentation}
Nous avons utilisé Doxygen pour gérer la documentation de nos fonctions. Elle est accessible par \texttt{doc/html/index.html} une fois la commande "\texttt{make doc}" effectuée.

\subsection{Tests unitaires}
Nous utilisons CuTest pour tester nos fonctions. Ils sont regroupés et exécutés automatiquement lorsque la commande "\texttt{make test}" est utilisée.

\subsection{CMake}
CMake génère un makefile et vérifie la configuration du système.

\subsection{Mozilla Firefox 37.2}
Nous avons créé des profils différents et un script qui ouvre le profil de test.
\subsubsection{Création d'un profil}
Lancez Firefox via la commande "\texttt{firefox -new-instance -P}". Cela ouvrira le gestionnaire de profils. Cliquez sur "\texttt{Create profile}", nommez-le et lancez Firefox avec.
\subsubsection{Configuration de Firefox}
Dans "Menu","Préférences","Avancé", naviguez dans l'onglet "Réseau" puis cliquez sur "Paramètres". Sélectionnez ensuite "Configuration manuelle du proxy". Dans "Proxy HTTP", entrez localhost et dans "Port" 8080.

\subsection{Testé sous Ubuntu et Debian}
Devrait fonctionner sous tout système UNIX avec sockets.
